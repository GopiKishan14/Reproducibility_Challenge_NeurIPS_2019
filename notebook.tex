
% Default to the notebook output style

    


% Inherit from the specified cell style.




    
\documentclass[11pt]{article}

    
    
    \usepackage[T1]{fontenc}
    % Nicer default font (+ math font) than Computer Modern for most use cases
    \usepackage{mathpazo}

    % Basic figure setup, for now with no caption control since it's done
    % automatically by Pandoc (which extracts ![](path) syntax from Markdown).
    \usepackage{graphicx}
    % We will generate all images so they have a width \maxwidth. This means
    % that they will get their normal width if they fit onto the page, but
    % are scaled down if they would overflow the margins.
    \makeatletter
    \def\maxwidth{\ifdim\Gin@nat@width>\linewidth\linewidth
    \else\Gin@nat@width\fi}
    \makeatother
    \let\Oldincludegraphics\includegraphics
    % Set max figure width to be 80% of text width, for now hardcoded.
    \renewcommand{\includegraphics}[1]{\Oldincludegraphics[width=.8\maxwidth]{#1}}
    % Ensure that by default, figures have no caption (until we provide a
    % proper Figure object with a Caption API and a way to capture that
    % in the conversion process - todo).
    \usepackage{caption}
    \DeclareCaptionLabelFormat{nolabel}{}
    \captionsetup{labelformat=nolabel}

    \usepackage{adjustbox} % Used to constrain images to a maximum size 
    \usepackage{xcolor} % Allow colors to be defined
    \usepackage{enumerate} % Needed for markdown enumerations to work
    \usepackage{geometry} % Used to adjust the document margins
    \usepackage{amsmath} % Equations
    \usepackage{amssymb} % Equations
    \usepackage{textcomp} % defines textquotesingle
    % Hack from http://tex.stackexchange.com/a/47451/13684:
    \AtBeginDocument{%
        \def\PYZsq{\textquotesingle}% Upright quotes in Pygmentized code
    }
    \usepackage{upquote} % Upright quotes for verbatim code
    \usepackage{eurosym} % defines \euro
    \usepackage[mathletters]{ucs} % Extended unicode (utf-8) support
    \usepackage[utf8x]{inputenc} % Allow utf-8 characters in the tex document
    \usepackage{fancyvrb} % verbatim replacement that allows latex
    \usepackage{grffile} % extends the file name processing of package graphics 
                         % to support a larger range 
    % The hyperref package gives us a pdf with properly built
    % internal navigation ('pdf bookmarks' for the table of contents,
    % internal cross-reference links, web links for URLs, etc.)
    \usepackage{hyperref}
    \usepackage{longtable} % longtable support required by pandoc >1.10
    \usepackage{booktabs}  % table support for pandoc > 1.12.2
    \usepackage[inline]{enumitem} % IRkernel/repr support (it uses the enumerate* environment)
    \usepackage[normalem]{ulem} % ulem is needed to support strikethroughs (\sout)
                                % normalem makes italics be italics, not underlines
    

    
    
    % Colors for the hyperref package
    \definecolor{urlcolor}{rgb}{0,.145,.698}
    \definecolor{linkcolor}{rgb}{.71,0.21,0.01}
    \definecolor{citecolor}{rgb}{.12,.54,.11}

    % ANSI colors
    \definecolor{ansi-black}{HTML}{3E424D}
    \definecolor{ansi-black-intense}{HTML}{282C36}
    \definecolor{ansi-red}{HTML}{E75C58}
    \definecolor{ansi-red-intense}{HTML}{B22B31}
    \definecolor{ansi-green}{HTML}{00A250}
    \definecolor{ansi-green-intense}{HTML}{007427}
    \definecolor{ansi-yellow}{HTML}{DDB62B}
    \definecolor{ansi-yellow-intense}{HTML}{B27D12}
    \definecolor{ansi-blue}{HTML}{208FFB}
    \definecolor{ansi-blue-intense}{HTML}{0065CA}
    \definecolor{ansi-magenta}{HTML}{D160C4}
    \definecolor{ansi-magenta-intense}{HTML}{A03196}
    \definecolor{ansi-cyan}{HTML}{60C6C8}
    \definecolor{ansi-cyan-intense}{HTML}{258F8F}
    \definecolor{ansi-white}{HTML}{C5C1B4}
    \definecolor{ansi-white-intense}{HTML}{A1A6B2}

    % commands and environments needed by pandoc snippets
    % extracted from the output of `pandoc -s`
    \providecommand{\tightlist}{%
      \setlength{\itemsep}{0pt}\setlength{\parskip}{0pt}}
    \DefineVerbatimEnvironment{Highlighting}{Verbatim}{commandchars=\\\{\}}
    % Add ',fontsize=\small' for more characters per line
    \newenvironment{Shaded}{}{}
    \newcommand{\KeywordTok}[1]{\textcolor[rgb]{0.00,0.44,0.13}{\textbf{{#1}}}}
    \newcommand{\DataTypeTok}[1]{\textcolor[rgb]{0.56,0.13,0.00}{{#1}}}
    \newcommand{\DecValTok}[1]{\textcolor[rgb]{0.25,0.63,0.44}{{#1}}}
    \newcommand{\BaseNTok}[1]{\textcolor[rgb]{0.25,0.63,0.44}{{#1}}}
    \newcommand{\FloatTok}[1]{\textcolor[rgb]{0.25,0.63,0.44}{{#1}}}
    \newcommand{\CharTok}[1]{\textcolor[rgb]{0.25,0.44,0.63}{{#1}}}
    \newcommand{\StringTok}[1]{\textcolor[rgb]{0.25,0.44,0.63}{{#1}}}
    \newcommand{\CommentTok}[1]{\textcolor[rgb]{0.38,0.63,0.69}{\textit{{#1}}}}
    \newcommand{\OtherTok}[1]{\textcolor[rgb]{0.00,0.44,0.13}{{#1}}}
    \newcommand{\AlertTok}[1]{\textcolor[rgb]{1.00,0.00,0.00}{\textbf{{#1}}}}
    \newcommand{\FunctionTok}[1]{\textcolor[rgb]{0.02,0.16,0.49}{{#1}}}
    \newcommand{\RegionMarkerTok}[1]{{#1}}
    \newcommand{\ErrorTok}[1]{\textcolor[rgb]{1.00,0.00,0.00}{\textbf{{#1}}}}
    \newcommand{\NormalTok}[1]{{#1}}
    
    % Additional commands for more recent versions of Pandoc
    \newcommand{\ConstantTok}[1]{\textcolor[rgb]{0.53,0.00,0.00}{{#1}}}
    \newcommand{\SpecialCharTok}[1]{\textcolor[rgb]{0.25,0.44,0.63}{{#1}}}
    \newcommand{\VerbatimStringTok}[1]{\textcolor[rgb]{0.25,0.44,0.63}{{#1}}}
    \newcommand{\SpecialStringTok}[1]{\textcolor[rgb]{0.73,0.40,0.53}{{#1}}}
    \newcommand{\ImportTok}[1]{{#1}}
    \newcommand{\DocumentationTok}[1]{\textcolor[rgb]{0.73,0.13,0.13}{\textit{{#1}}}}
    \newcommand{\AnnotationTok}[1]{\textcolor[rgb]{0.38,0.63,0.69}{\textbf{\textit{{#1}}}}}
    \newcommand{\CommentVarTok}[1]{\textcolor[rgb]{0.38,0.63,0.69}{\textbf{\textit{{#1}}}}}
    \newcommand{\VariableTok}[1]{\textcolor[rgb]{0.10,0.09,0.49}{{#1}}}
    \newcommand{\ControlFlowTok}[1]{\textcolor[rgb]{0.00,0.44,0.13}{\textbf{{#1}}}}
    \newcommand{\OperatorTok}[1]{\textcolor[rgb]{0.40,0.40,0.40}{{#1}}}
    \newcommand{\BuiltInTok}[1]{{#1}}
    \newcommand{\ExtensionTok}[1]{{#1}}
    \newcommand{\PreprocessorTok}[1]{\textcolor[rgb]{0.74,0.48,0.00}{{#1}}}
    \newcommand{\AttributeTok}[1]{\textcolor[rgb]{0.49,0.56,0.16}{{#1}}}
    \newcommand{\InformationTok}[1]{\textcolor[rgb]{0.38,0.63,0.69}{\textbf{\textit{{#1}}}}}
    \newcommand{\WarningTok}[1]{\textcolor[rgb]{0.38,0.63,0.69}{\textbf{\textit{{#1}}}}}
    
    
    % Define a nice break command that doesn't care if a line doesn't already
    % exist.
    \def\br{\hspace*{\fill} \\* }
    % Math Jax compatability definitions
    \def\gt{>}
    \def\lt{<}
    % Document parameters
    \title{README}
    
    
    

    % Pygments definitions
    
\makeatletter
\def\PY@reset{\let\PY@it=\relax \let\PY@bf=\relax%
    \let\PY@ul=\relax \let\PY@tc=\relax%
    \let\PY@bc=\relax \let\PY@ff=\relax}
\def\PY@tok#1{\csname PY@tok@#1\endcsname}
\def\PY@toks#1+{\ifx\relax#1\empty\else%
    \PY@tok{#1}\expandafter\PY@toks\fi}
\def\PY@do#1{\PY@bc{\PY@tc{\PY@ul{%
    \PY@it{\PY@bf{\PY@ff{#1}}}}}}}
\def\PY#1#2{\PY@reset\PY@toks#1+\relax+\PY@do{#2}}

\expandafter\def\csname PY@tok@w\endcsname{\def\PY@tc##1{\textcolor[rgb]{0.73,0.73,0.73}{##1}}}
\expandafter\def\csname PY@tok@c\endcsname{\let\PY@it=\textit\def\PY@tc##1{\textcolor[rgb]{0.25,0.50,0.50}{##1}}}
\expandafter\def\csname PY@tok@cp\endcsname{\def\PY@tc##1{\textcolor[rgb]{0.74,0.48,0.00}{##1}}}
\expandafter\def\csname PY@tok@k\endcsname{\let\PY@bf=\textbf\def\PY@tc##1{\textcolor[rgb]{0.00,0.50,0.00}{##1}}}
\expandafter\def\csname PY@tok@kp\endcsname{\def\PY@tc##1{\textcolor[rgb]{0.00,0.50,0.00}{##1}}}
\expandafter\def\csname PY@tok@kt\endcsname{\def\PY@tc##1{\textcolor[rgb]{0.69,0.00,0.25}{##1}}}
\expandafter\def\csname PY@tok@o\endcsname{\def\PY@tc##1{\textcolor[rgb]{0.40,0.40,0.40}{##1}}}
\expandafter\def\csname PY@tok@ow\endcsname{\let\PY@bf=\textbf\def\PY@tc##1{\textcolor[rgb]{0.67,0.13,1.00}{##1}}}
\expandafter\def\csname PY@tok@nb\endcsname{\def\PY@tc##1{\textcolor[rgb]{0.00,0.50,0.00}{##1}}}
\expandafter\def\csname PY@tok@nf\endcsname{\def\PY@tc##1{\textcolor[rgb]{0.00,0.00,1.00}{##1}}}
\expandafter\def\csname PY@tok@nc\endcsname{\let\PY@bf=\textbf\def\PY@tc##1{\textcolor[rgb]{0.00,0.00,1.00}{##1}}}
\expandafter\def\csname PY@tok@nn\endcsname{\let\PY@bf=\textbf\def\PY@tc##1{\textcolor[rgb]{0.00,0.00,1.00}{##1}}}
\expandafter\def\csname PY@tok@ne\endcsname{\let\PY@bf=\textbf\def\PY@tc##1{\textcolor[rgb]{0.82,0.25,0.23}{##1}}}
\expandafter\def\csname PY@tok@nv\endcsname{\def\PY@tc##1{\textcolor[rgb]{0.10,0.09,0.49}{##1}}}
\expandafter\def\csname PY@tok@no\endcsname{\def\PY@tc##1{\textcolor[rgb]{0.53,0.00,0.00}{##1}}}
\expandafter\def\csname PY@tok@nl\endcsname{\def\PY@tc##1{\textcolor[rgb]{0.63,0.63,0.00}{##1}}}
\expandafter\def\csname PY@tok@ni\endcsname{\let\PY@bf=\textbf\def\PY@tc##1{\textcolor[rgb]{0.60,0.60,0.60}{##1}}}
\expandafter\def\csname PY@tok@na\endcsname{\def\PY@tc##1{\textcolor[rgb]{0.49,0.56,0.16}{##1}}}
\expandafter\def\csname PY@tok@nt\endcsname{\let\PY@bf=\textbf\def\PY@tc##1{\textcolor[rgb]{0.00,0.50,0.00}{##1}}}
\expandafter\def\csname PY@tok@nd\endcsname{\def\PY@tc##1{\textcolor[rgb]{0.67,0.13,1.00}{##1}}}
\expandafter\def\csname PY@tok@s\endcsname{\def\PY@tc##1{\textcolor[rgb]{0.73,0.13,0.13}{##1}}}
\expandafter\def\csname PY@tok@sd\endcsname{\let\PY@it=\textit\def\PY@tc##1{\textcolor[rgb]{0.73,0.13,0.13}{##1}}}
\expandafter\def\csname PY@tok@si\endcsname{\let\PY@bf=\textbf\def\PY@tc##1{\textcolor[rgb]{0.73,0.40,0.53}{##1}}}
\expandafter\def\csname PY@tok@se\endcsname{\let\PY@bf=\textbf\def\PY@tc##1{\textcolor[rgb]{0.73,0.40,0.13}{##1}}}
\expandafter\def\csname PY@tok@sr\endcsname{\def\PY@tc##1{\textcolor[rgb]{0.73,0.40,0.53}{##1}}}
\expandafter\def\csname PY@tok@ss\endcsname{\def\PY@tc##1{\textcolor[rgb]{0.10,0.09,0.49}{##1}}}
\expandafter\def\csname PY@tok@sx\endcsname{\def\PY@tc##1{\textcolor[rgb]{0.00,0.50,0.00}{##1}}}
\expandafter\def\csname PY@tok@m\endcsname{\def\PY@tc##1{\textcolor[rgb]{0.40,0.40,0.40}{##1}}}
\expandafter\def\csname PY@tok@gh\endcsname{\let\PY@bf=\textbf\def\PY@tc##1{\textcolor[rgb]{0.00,0.00,0.50}{##1}}}
\expandafter\def\csname PY@tok@gu\endcsname{\let\PY@bf=\textbf\def\PY@tc##1{\textcolor[rgb]{0.50,0.00,0.50}{##1}}}
\expandafter\def\csname PY@tok@gd\endcsname{\def\PY@tc##1{\textcolor[rgb]{0.63,0.00,0.00}{##1}}}
\expandafter\def\csname PY@tok@gi\endcsname{\def\PY@tc##1{\textcolor[rgb]{0.00,0.63,0.00}{##1}}}
\expandafter\def\csname PY@tok@gr\endcsname{\def\PY@tc##1{\textcolor[rgb]{1.00,0.00,0.00}{##1}}}
\expandafter\def\csname PY@tok@ge\endcsname{\let\PY@it=\textit}
\expandafter\def\csname PY@tok@gs\endcsname{\let\PY@bf=\textbf}
\expandafter\def\csname PY@tok@gp\endcsname{\let\PY@bf=\textbf\def\PY@tc##1{\textcolor[rgb]{0.00,0.00,0.50}{##1}}}
\expandafter\def\csname PY@tok@go\endcsname{\def\PY@tc##1{\textcolor[rgb]{0.53,0.53,0.53}{##1}}}
\expandafter\def\csname PY@tok@gt\endcsname{\def\PY@tc##1{\textcolor[rgb]{0.00,0.27,0.87}{##1}}}
\expandafter\def\csname PY@tok@err\endcsname{\def\PY@bc##1{\setlength{\fboxsep}{0pt}\fcolorbox[rgb]{1.00,0.00,0.00}{1,1,1}{\strut ##1}}}
\expandafter\def\csname PY@tok@kc\endcsname{\let\PY@bf=\textbf\def\PY@tc##1{\textcolor[rgb]{0.00,0.50,0.00}{##1}}}
\expandafter\def\csname PY@tok@kd\endcsname{\let\PY@bf=\textbf\def\PY@tc##1{\textcolor[rgb]{0.00,0.50,0.00}{##1}}}
\expandafter\def\csname PY@tok@kn\endcsname{\let\PY@bf=\textbf\def\PY@tc##1{\textcolor[rgb]{0.00,0.50,0.00}{##1}}}
\expandafter\def\csname PY@tok@kr\endcsname{\let\PY@bf=\textbf\def\PY@tc##1{\textcolor[rgb]{0.00,0.50,0.00}{##1}}}
\expandafter\def\csname PY@tok@bp\endcsname{\def\PY@tc##1{\textcolor[rgb]{0.00,0.50,0.00}{##1}}}
\expandafter\def\csname PY@tok@fm\endcsname{\def\PY@tc##1{\textcolor[rgb]{0.00,0.00,1.00}{##1}}}
\expandafter\def\csname PY@tok@vc\endcsname{\def\PY@tc##1{\textcolor[rgb]{0.10,0.09,0.49}{##1}}}
\expandafter\def\csname PY@tok@vg\endcsname{\def\PY@tc##1{\textcolor[rgb]{0.10,0.09,0.49}{##1}}}
\expandafter\def\csname PY@tok@vi\endcsname{\def\PY@tc##1{\textcolor[rgb]{0.10,0.09,0.49}{##1}}}
\expandafter\def\csname PY@tok@vm\endcsname{\def\PY@tc##1{\textcolor[rgb]{0.10,0.09,0.49}{##1}}}
\expandafter\def\csname PY@tok@sa\endcsname{\def\PY@tc##1{\textcolor[rgb]{0.73,0.13,0.13}{##1}}}
\expandafter\def\csname PY@tok@sb\endcsname{\def\PY@tc##1{\textcolor[rgb]{0.73,0.13,0.13}{##1}}}
\expandafter\def\csname PY@tok@sc\endcsname{\def\PY@tc##1{\textcolor[rgb]{0.73,0.13,0.13}{##1}}}
\expandafter\def\csname PY@tok@dl\endcsname{\def\PY@tc##1{\textcolor[rgb]{0.73,0.13,0.13}{##1}}}
\expandafter\def\csname PY@tok@s2\endcsname{\def\PY@tc##1{\textcolor[rgb]{0.73,0.13,0.13}{##1}}}
\expandafter\def\csname PY@tok@sh\endcsname{\def\PY@tc##1{\textcolor[rgb]{0.73,0.13,0.13}{##1}}}
\expandafter\def\csname PY@tok@s1\endcsname{\def\PY@tc##1{\textcolor[rgb]{0.73,0.13,0.13}{##1}}}
\expandafter\def\csname PY@tok@mb\endcsname{\def\PY@tc##1{\textcolor[rgb]{0.40,0.40,0.40}{##1}}}
\expandafter\def\csname PY@tok@mf\endcsname{\def\PY@tc##1{\textcolor[rgb]{0.40,0.40,0.40}{##1}}}
\expandafter\def\csname PY@tok@mh\endcsname{\def\PY@tc##1{\textcolor[rgb]{0.40,0.40,0.40}{##1}}}
\expandafter\def\csname PY@tok@mi\endcsname{\def\PY@tc##1{\textcolor[rgb]{0.40,0.40,0.40}{##1}}}
\expandafter\def\csname PY@tok@il\endcsname{\def\PY@tc##1{\textcolor[rgb]{0.40,0.40,0.40}{##1}}}
\expandafter\def\csname PY@tok@mo\endcsname{\def\PY@tc##1{\textcolor[rgb]{0.40,0.40,0.40}{##1}}}
\expandafter\def\csname PY@tok@ch\endcsname{\let\PY@it=\textit\def\PY@tc##1{\textcolor[rgb]{0.25,0.50,0.50}{##1}}}
\expandafter\def\csname PY@tok@cm\endcsname{\let\PY@it=\textit\def\PY@tc##1{\textcolor[rgb]{0.25,0.50,0.50}{##1}}}
\expandafter\def\csname PY@tok@cpf\endcsname{\let\PY@it=\textit\def\PY@tc##1{\textcolor[rgb]{0.25,0.50,0.50}{##1}}}
\expandafter\def\csname PY@tok@c1\endcsname{\let\PY@it=\textit\def\PY@tc##1{\textcolor[rgb]{0.25,0.50,0.50}{##1}}}
\expandafter\def\csname PY@tok@cs\endcsname{\let\PY@it=\textit\def\PY@tc##1{\textcolor[rgb]{0.25,0.50,0.50}{##1}}}

\def\PYZbs{\char`\\}
\def\PYZus{\char`\_}
\def\PYZob{\char`\{}
\def\PYZcb{\char`\}}
\def\PYZca{\char`\^}
\def\PYZam{\char`\&}
\def\PYZlt{\char`\<}
\def\PYZgt{\char`\>}
\def\PYZsh{\char`\#}
\def\PYZpc{\char`\%}
\def\PYZdl{\char`\$}
\def\PYZhy{\char`\-}
\def\PYZsq{\char`\'}
\def\PYZdq{\char`\"}
\def\PYZti{\char`\~}
% for compatibility with earlier versions
\def\PYZat{@}
\def\PYZlb{[}
\def\PYZrb{]}
\makeatother


    % Exact colors from NB
    \definecolor{incolor}{rgb}{0.0, 0.0, 0.5}
    \definecolor{outcolor}{rgb}{0.545, 0.0, 0.0}



    
    % Prevent overflowing lines due to hard-to-break entities
    \sloppy 
    % Setup hyperref package
    \hypersetup{
      breaklinks=true,  % so long urls are correctly broken across lines
      colorlinks=true,
      urlcolor=urlcolor,
      linkcolor=linkcolor,
      citecolor=citecolor,
      }
    % Slightly bigger margins than the latex defaults
    
    \geometry{verbose,tmargin=1in,bmargin=1in,lmargin=1in,rmargin=1in}
    
    

    \begin{document}
    
    
    \maketitle
    
    

    
    \section{Reproducibility\_Challenge\_NeurIPS\_2019}\label{reproducibility_challenge_neurips_2019}

The report is present
\href{https://gopikishan14.github.io/Reproducibility_Challenge_NeurIPS_2019/}{here}

\section{Abstract}\label{abstract}

This is a report for reproducibility challenge of NeurlIPS'19 on the
paper \texttt{Competitive\ gradient\ descent}
\href{https://arxiv.org/abs/1905.12103}{(Schäfer et al., 2019)}. The
paper introduces a novel algorithm for the numerical computation of Nash
equilibria of competitive two-player games. It avoids oscillatory and
divergent behaviors seen in alternating gradient descent. The paper
proposes several experiments to establish the robustness of their
method. This project aims at replicating their results.

This report also summerises the method proposed in the original paper.
Refer to the report for details. I would like to summerize the method
here.

    \section{Paper Overview}\label{paper-overview}

The paper introduces a new algorithm for the numerical computation of
Nash equilibria of competitive two-player games. The method is a natural
generalization of gradient descent to the two-player setting where the
update is given by the Nash equilibrium of a regularized bilinear local
approximation of the underlying game. It avoids oscillatory and
divergent behaviors seen in alternating gradient descent. Convergence
and stability properties of the method are robust to strong interactions
between the players, without adapting the stepsize, which is not the
case with previous methods. The ability to choose larger stepsizes
furthermore allows the algorithm to achieve faster convergence, as
measured by the number of model evaluations (See the
\href{https://gopikishan14.github.io/Reproducibility_Challenge_NeurIPS_2019/}{report}
experiments section).

    \subsection{Background}\label{background}

The traditional optimization is concerned with a single agent trying to
optimize a cost function. It can be seen as \(\min_{x \in R^m} f(x)\) .
The agent has a clear objective to find (``Good local'') minimum of f.
Gradeint Descent (and its varients) are reliable Algorithmic Baseline
for this purpose.

The paper talks about Competitive optimization. Competitive optimization
extends this problem to the setting of multiple agents each trying to
minimize their own cost function, which in general depends on the
actions of all agents. The paper deals with the case of two such agents:

\begin{align}
        &\min_{x \in R^m} f(x,y),\ \ \ \min_{y \in R^n} g(x,y)
    \end{align}

for two functions \(f,g: R^m \times R^n \longrightarrow R\).

In single agent optimization, the solution of the problem consists of
the minimizer of the cost function. In competitive optimization, the
right definition of solution is less obvious, but often one is
interested in computing Nash-- or strategic equilibria: Pairs of
strategies, such that no player can decrease their costs by unilaterally
changing their strategies. If f and g are not convex, finding a global
Nash equilibrium is typically impossible and instead we hope to find a
"good" local Nash equilibrium

    \subsection{About the problem}\label{about-the-problem}

\subparagraph{Gradient descent/ascent and the cycling
problem:}\label{gradient-descentascent-and-the-cycling-problem}

For differentiable objective functions, the most naive approach to
solving

\begin{align}
        \label{eqn:game}
        &\min_{x \in R^m} f(x,y),\ \ \ \min_{y \in R^n} g(x,y)
    \end{align}

is gradient descent ascent (GDA), whereby both players independently
change their strategy in the direction of steepest descent of their cost
function. Unfortunately, this procedure features oscillatory or
divergent behavior even in the simple case of a bilinear game
(\(f(x,y) = x^{\top} y = -g(x,y)\))

    \subsection{Solution approach}\label{solution-approach}

To motivate this algorithm, authors remind us that gradient descent with
stepsize \(\eta\) applied to the function \(f:R^m \longrightarrow R\)
can be written as

\begin{equation}
        x_{k+1} = argmin_{x \in R^m} (x^T - x_{k}^T) \nabla_x f(x_k) + \frac{1}{2\eta} \|x - x_{k}\|^2.
    \end{equation}

This models a (single) player solving a local linear approximation of
the (minimization) game, subject to a quadratic penalty that expresses
her limited confidence in the global accuracy of the model.

\texttt{The\ natural\ generalization\ of\ this\ idea\ to\ the\ competitive\ case\ should\ then\ be\ given\ by\ the\ two\ players\ solving\ a\ local\ approximation\ of\ the\ true\ game,\ both\ subject\ to\ a\ quadratic\ penalty\ that\ expresses\ their\ limited\ confidence\ in\ the\ accuracy\ of\ the\ local\ approximation.}

In order to implement this idea, we need to find the appropriate way to
generalize the linear approximation in the single agent setting to the
competitive setting.

Authors suggest to use a \textbf{bilinear} approximation in the
two-player setting. Since the bilinear approximation is the lowest order
approximation that can capture some interaction between the two players,
they argue that the natural generalization of gradient descent to
competitive optimization is not GDA, but rather the update rule
\((x_{k+1},y_{k+1}) = (x_k,y_k) + (x,y)\), where \((x,y)\) is a Nash
equilibrium of \textbf{the game}.

\begin{align}
    \begin{split}
    \label{eqn:localgame}
    \min_{x \in R^m} x^{\top} \nabla_x f &+ x^{\top} D_{xy}^2 f y + y^{\top} \nabla_y f + \frac{1}{2\eta} x^{\top} x \\
    \min_{y \in R^n}  y^{\top} \nabla_y g &+ y^{\top} D_{yx}^2 g x + x^{\top} \nabla_x g + \frac{1}{2\eta} y^{\top} y.
    \end{split}
\end{align}

Indeed, the (unique) Nash equilibrium of the above Game can be computed
in closed form.

    \subsection{Proposed method}\label{proposed-method}

**Among all (possibly randomized) strategies with finite first moment,
the only Nash equilibrium of \texttt{the\ Game} is given by

\begin{align}
\label{eqn:nash}
&x = -\eta \left( Id - \eta^2 D_{xy}^2f D_{yx}^2 g \right)^{-1}  
            \left( \nabla_{x} f - \eta D_{xy}^2f  \nabla_{y} g \right) \\
&y = -\eta \left( Id - \eta^2 D_{yx}^2g D_{xy}^2 f \right)^{-1}  
            \left( \nabla_{y} g - \eta D_{yx}^2g  \nabla_{x} f \right),
\end{align}

given that the matrix inverses in the above expression exist.**

Note that the matrix inverses exist for all but one value of \(\eta\),
and for all \(\eta\) in the case of a zero sum game.

According to the above Theorem, the Game has exactly one optimal pair of
strategies, which is deter-ministic. Thus, we can use these strategies
as an update rule, generalizing the idea of local optimalityfrom the
single-- to the multi agent setting and obtaining the following
Algorithm.

\texttt{Competitive\ Gradient\ Descent\ (CGD)}

\begin{align}
for\ (0 <= k <= N-1)\\
&x_{k+1}  = x_{k} - \eta \left( Id - \eta^2 D_{xy}^2f D_{yx}^2 g \right)^{-1}\left( \nabla_{x} f - \eta D_{xy}^2f  \nabla_{y} g \right)\\
&y_{k+1} = y_{k} - \eta \left( Id - \eta^2 D_{yx}^2g D_{xy}^2 f \right)^{-1}  
    \left( \nabla_{y} g - \eta D_{yx}^2g  \nabla_{x} f \right)\\
    return\ (x_{N},y_{N})\;
\end{align}

\textbf{What I think that they think that I think ... that they do}:
Another game-theoretic interpretation of CGD follows from the
observation that its update rule can be written as

\begin{equation}
\label{eqn:whatIthink}
    \begin{pmatrix}
         \Delta x\\
         \Delta y
    \end{pmatrix}
    =  
    -
    \begin{pmatrix}
        Id        & \eta D_{xy}^2 f \\
        \eta D_{yx}^2 g & Id        
    \end{pmatrix}^{-1}
    \begin{pmatrix}
         \nabla_{x} f\\
         \nabla_{y} g
    \end{pmatrix}.
\end{equation}

Applying the expansion \$ \lambda\emph{\{\max\} (A) \textless{} 1
\Rightarrow \left( Id - A \right)\^{}\{-1\} = \lim}\{N
\rightarrow \infty\} \sum\_\{k=0\}\^{}\{N\} A\^{}k\$ to the above
equation, we observe that: \textbackslash{}

\begin{enumerate}
\def\labelenumi{\arabic{enumi}.}
\tightlist
\item
  The first partial sum (\(N = 0\)) corresponds to the optimal strategy
  if the other player's strategy stays constant (GDA).
\item
  The second partial sum (\(N = 1\)) corresponds to the optimal strategy
  if the other player thinks that the other player's strategy stays
  constant (LCGD).
\item
  The third partial sum (\(N = 2\)) corresponds to the optimal strategy
  if the other player thinks that the other player thinks that the other
  player's strategy stays constant, and so forth, until the Nash
  equilibrium is recovered in the limit.
\end{enumerate}

    \subsection{Comparison}\label{comparison}

These six algorithms amount to different subsets of the following four
terms.

\begin{align*}
       & \text{GDA: } &\Delta x =  &&&- \nabla_x f&\\
       & \text{LCGD: } &\Delta x =  &&&- \nabla_x f& &-\eta D_{xy}^2 f \nabla_y f&\\
       & \text{SGA: } &\Delta x =  &&&- \nabla_x f& &- \gamma D_{xy}^2 f \nabla_y f&  & & \\
       & \text{ConOpt: } &\Delta x =  &&&- \nabla_x f& &- \gamma D_{xy}^2 f \nabla_y f&  &- \gamma D_{xx}^2 f \nabla_x f& \\
       & \text{OGDA: } &\Delta x \approx &&&- \nabla_x f& &-\eta D_{xy}^2 f \nabla_y f&  &+\eta D_{xx}^2 f \nabla_x f& \\
       & \text{CGD: } &\Delta x = &\left(Id + \eta^2 D_{xy}^2 f D_{yx}^2 f\right)^{-1}&\bigl( &- \nabla_x f&  &-\eta D_{xy}^2 f \nabla_y f& & & \bigr)
     \end{align*}

\begin{enumerate}
\def\labelenumi{\arabic{enumi}.}
\item
  The \textbf{gradient term} \(-\nabla_{x}f\), \(\nabla_{y}f\) which
  corresponds to the most immediate way in which the players can improve
  their cost.
\item
  The \textbf{competitive term} \(-D_{xy}f \nabla_yf\),
  \(D_{yx}f \nabla_x f\) which can be interpreted either as anticipating
  the other player to use the naive (GDA) strategy, or as decreasing the
  other players influence (by decreasing their gradient).
\item
  The \textbf{consensus term} \$ \pm D\_\{xx\}\^{}2 \nabla\_x f\$,
  \(\mp D_{yy}^2 \nabla_y f\) that determines whether the players prefer
  to decrease their gradient (\(\pm = +\)) or to increase it
  (\(\pm = -\)). The former corresponds the players seeking consensus,
  whereas the latter can be seen as the opposite of consensus. (It also
  corresponds to an approximate Newton's method.
  \textbackslash{}footnote\{Applying a damped and regularized Newton's
  method to the optimization problem of Player 1 would amount to
  choosing
  \(x_{k+1} = x_{k} - \eta(Id + \eta D_{xx}^2)^{-1} f \nabla_x f \approx x_{k} - \eta( \nabla_xf - \eta D_{xx}^{2}f \nabla_x f)\),
  for \(\|\eta D_{xx}^2f\| \ll 1\).)
\item
  The \textbf{equilibrium term}
  \((Id + \eta^2 D_{xy}^2 D_{yx}^2 f)^{-1}\),
  \((Id + \eta^2 D_{yx}^2 D_{xy}^2 f)^{-1}\), which arises from the
  players solving for the Nash equilibrium. This term lets each player
  prefer strategies that are less vulnerable to the actions of the other
  player.
\end{enumerate}

    \subsection{Code Implementation}\label{code-implementation}

\subsubsection{Computing Hessian vector
products}\label{computing-hessian-vector-products}

\subsubsection{Matrix inversion for the equilibrium
term}\label{matrix-inversion-for-the-equilibrium-term}

    \subsection{Conclusion}\label{conclusion}

Look out to experiments and conclusion section


    % Add a bibliography block to the postdoc
    
    
    
    \end{document}
